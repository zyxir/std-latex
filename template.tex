%% 文檔類設定,本文檔類別爲「書籍」。
\documentclass[
% 書籍的排版默認開啓了 twoside(即頁面分奇偶,奇數頁和偶數頁的文字偏移、頁眉頁腳
% 位置均是對稱的)選項,但爲了方便電子閱讀,本文檔使用不分奇偶頁的 oneside 選項。
oneside,
% 設置默認字體大小爲 12 磅,對應中文字號小四。
12pt,
]{book}

%% 大部分設置放在了 zylatex.sty 包中,以便複用。
\usepackage[
style = en,
fontset = zytc,
layout = paper,
datetime = iso,
]{zylatex}

%% 多語言混排設置。
% zylatex 不可能知道您的文檔中用到了哪幾種語言,更不會知道您是否需要 polyglossia
% 所提供的跟據音節拆詞的功能,因此這一部分沒有集成進 zylatex,必須自己手動調用。
% 通常只包含中文和英文的文檔是不需要調用 polyglossia 的。
\usepackage{polyglossia}
\setdefaultlanguage{english}
\setotherlanguages{russian}

%% 頁眉頁腳設置。
\usepackage{fancyhdr}
\pagestyle{fancy}
\fancyhf{}
\fancyhead[LE, LO]{\LaTeX 模板文檔}
\fancyhead[RE, RO]{\leftmark}
\fancyfoot[CE, CO]{\thepage}

%% 封面、作者設置。
\title{
  \Huge \LaTeX 模板文檔 \\
  \Large \LaTeX\ Template Document
}
\author{
  陳卓 (Eric Chen)
}
\date{\DTMtoday}

\begin{document}

\maketitle

\tableofcontents

\chapter{模板介紹}

\LaTeX 是一種經常被用到的文檔製作工具。在平時的使用中,筆者發現,每次創建一個新的 LaTeX 項目時,都會有大量代碼需要被重複用到,比如:中文支持的開啓與配置,字體的搭配,頁面佈局的設置,常用包的啓用,表格的外觀配置,等等。因此,筆者把\textbf{自己經常用到的設置}整合了起來,製作成了這個名爲 \texttt{zylatex.sty} 的宏包,方便自己的日常使用。\textit{對於第三方用戶而言,此包可能包含大量筆者的個人偏好,並不完全符合用戶心中「好」的定義;因此,此宏包僅供參考,也歡迎任何人以此包作爲基礎開發符合自己偏好的包。}筆者個人使用此包的方式有:

\begin{enumerate}
\item 在小論文、說明文檔等 LaTeX 文檔中調用此包,省去大量配置過程。
\item 將 \href{https://orgmode.org/}{Org Mode} 文件導出爲 PDF 時,用作風格模板。
\end{enumerate}

\section{調用此包}

此包的調用方式與其它 LaTeX 包無異,衹需在 TeX 文件的前言部分(即 \texttt{\textbackslash begin\{document\}} 語句之前的部分)寫下如下語句即可

\begin{verbatim}
\usepackage[
option1 = xxx,
option2 = xxx,
...
optionN = xxx,
]{/path/to/zylatex}
\end{verbatim}

其中,\texttt{optionX} 指的是 \texttt{zylatex} 提供的諸多用戶可選項。\texttt{/path/to/zylatex} 指的是 \texttt{zylatex.sty} 的路徑,它可以是絕對路徑,也可以是相對於您的 TeX 文件的相對路徑;如果您把 \texttt{zylatex.sty} 與您的 TeX 文件放在同一目錄下的話,便可以直接寫作 \texttt{\textbackslash usepackage[...]\{zylatex\}}。

關於此包提供的的各個用戶可選項的用法,請閱讀本模板文檔的各個章節。

\chapter{一般設置}

\section{風格}

\texttt{zylatex} 使用 \texttt{style} 參數來控制文檔的風格,該參數共有三種可能值:\texttt{en}、\texttt{tc} 與 \texttt{sc}。

\begin{itemize}
\item \texttt{style=en} 段首不縮進,章節圖表名稱等使用英文。
\item \texttt{style=tc} 段首縮進,章節圖表名稱等使用傳統漢字。
\item \texttt{style=sc} 段首縮進,章節圖表名稱等使用簡體漢字。
\end{itemize}

本模板文檔採用的是 \texttt{style=en} 風格。

\section{字體選取}

\texttt{zylatex} 使用 C\TeX 宏集\footnote{C\TeX 宏集是筆者所知的最好的 \LaTeX 中文解決方案。見其 \href{https://ctan.org/pkg/ctex}{CTAN 頁面}。}來提供中文支持。在使用 \texttt{zylatex} 時,可以通過指定 \texttt{fontset} 參數來指定文檔使用的字體。具體配置方式是:

\begin{itemize}
\item \texttt{fontset=zytc} (默認選項)使用筆者自己搭配的一套字體方案,如表 \ref{tab:fontset-fonts} 所示。必須要安裝了方案中的每一個字體才能正常編譯。
\item \texttt{fontset=ctex} (推薦選項)使用 C\TeX 自動檢測用戶使用的操作系統,配置相應的字體。具體請參見\href{http://mirrors.ctan.org/language/chinese/ctex/ctex.pdf}{C\TeX 宏集手册}第 6-7 頁。
\item \texttt{fontset=adobe|fandol|founder|mac|macnew|macold|ubuntu|windows|none|...} 指定 C\TeX 宏集加載的字庫。具體請參見\href{http://mirrors.ctan.org/language/chinese/ctex/ctex.pdf}{C\TeX 宏集手册}第 6-7 頁。
\item \texttt{fontset=none} 不配置中文字體,由用戶自己配置。具體請參見\href{http://mirrors.ctan.org/language/chinese/ctex/ctex.pdf}{C\TeX 宏集手册}第 6-7 頁。
\end{itemize}

\begin{table}[h!]
  \centering
  \begin{tabular}{cccc}
    \hline
     字體名稱 & 字體命令 & \texttt{windows} 方案 & \texttt{zytc} 方案 \\
    \hline
    宋體(中文正文) & \texttt{\textbackslash songti} & 宋体 & 一點明體 \\
    黑體 & \texttt{\textbackslash heiti} & 黑体 & 更紗黑體 CL \\
    仿宋體 & \texttt{\textbackslash fangsong} & 仿宋 & Adobe 仿宋 Std \\
    楷書 & \texttt{\textbackslash kaishu} & 楷书 & 文鼎PL中楷 \\
    衬綫體(西文正文) & 無 & \LaTeX 默認 & IBM Plex Serif \\
    無衬綫體 & \texttt{\textbackslash textsf} & \LaTeX 默認 & Roboto \\
    等寬體 & \texttt{\textbackslash texttt} & \LaTeX 默認 & Sarasa Mono HC \\
    \hline
  \end{tabular}
  \caption{C\TeX 的 \texttt{windows} 方案與筆者自用的 \texttt{zytc} 方案之對比}
  \label{tab:fontset-fonts}
\end{table}

\section{頁面設置}

\texttt{zylatex} 提供了兩種頁面佈局:\texttt{layout=paper} 和 \texttt{layout=phone},分別適合紙面打印和手機屏幕閱讀。\texttt{layout=phone} 會使用極小的頁邊距、較小的標題字號和較小的行距,確保在手機等移動設備的屏幕上也能看到足够內容。

\chapter{示例文本}

\section{中文示例}

(以下內容爲諸葛亮的《出師表》)

臣亮言:先帝創業未半,而中道崩殂。今天下三分,益州疲敝,此誠危急存亡之秋也!然侍衞之臣,不懈於內;忠志之士,忘身於外者,蓋追先帝之殊遇,欲報之於陛下也。誠宜開張聖聽,以光先帝遺德,恢弘志士之氣;不宜妄自菲薄,引喻失義,以塞忠諫之路也。

(黑體){\heiti 宮中、府中,俱為一體;陟罰臧否,不宜異同。若有作奸、犯科,及為忠善者,宜付有司,論其刑賞,以昭陛下平明之治;不宜偏私,使內外異法也。}

(仿宋體){\fangsong 侍中、侍郎郭攸之、費禕、董允等,此皆良實,志慮忠純,是以先帝簡拔以遺陛下。愚以為宮中之事,事無大小,悉以咨之,然後施行,必能裨補闕漏,有所廣益。將軍向寵,性行淑均,曉暢軍事,試用於昔日,先帝稱之曰「能」,是以眾議舉寵為督。愚以為營中之事,事無大小,悉以咨之,必能使行陣和睦,優劣得所也。}

(楷書){\kaishu 親賢臣,遠小人,此先漢所以興隆也;親小人,遠賢臣,此後漢所以傾頹也。先帝在時,每與臣論此事,未嘗不歎息痛恨於桓、靈也。侍中、尚書、長史、參軍,此悉貞亮死節之臣也,願陛下親之信之,則漢室之隆,可計日而待也。}

(粗體)\textbf{臣本布衣,躬耕於南陽,苟全性命於亂世,不求聞達於諸侯。先帝不以臣卑鄙,猥自枉屈,三顧臣於草廬之中,諮臣以當世之事;由是感激,遂許先帝以驅馳。後值傾覆,受任於敗軍之際,奉命於危難之間,爾來二十有一年矣!先帝知臣謹慎,故臨崩寄臣以大事也。受命以來,夙夜憂歎,恐託付不效,以傷先帝之明,故五月渡瀘,深入不毛。今南方已定,兵甲已足,當獎帥三軍,北定中原,庶竭駑鈍,攘除奸凶,興復漢室,還於舊都。此臣所以報先帝,而忠陛下之職分也。至於斟酌損益,進盡忠言,則攸之、禕、允之任也。}

願陛下託臣以討賊興復之效;不效,則治臣之罪,以告先帝之靈。若無興德之言,則責攸之、禕、允等之 咎,以彰其慢。陛下亦宜自謀,以諮諏善道,察納雅言,深追先帝遺詔。臣不勝受恩感激。今當遠離,臨表涕泣,不知所云。

\section{西文示例(拉丁字母)}

(以下內容節選自 Martin Luther King, Jr. 的 I Have a Dream)

I am happy to join with you today in what will go down in history as the greatest demonstration for freedom in the history of our nation.

(無衬綫體) \textsf{Five score years ago, a great American, in whose symbolic shadow we stand today, signed the Emancipation Proclamation. This momentous decree came as a great beacon light of hope to millions of Negro slaves who had been seared in the flames of withering injustice. It came as a joyous daybreak to end the long night of their captivity.}

(等寬體) \texttt{But one hundred years later, the Negro still is not free.} (粗體) \textbf{One hundred years later, the life of the Negro is still sadly crippled by the manacles of segregation and the chains of discrimination.} (斜體) \textit{One hundred years later, the Negro lives on a lonely island of poverty in the midst of a vast ocean of material prosperity.} One hundred years later, the Negro is still languished in the corners of American society and finds himself an exile in his own land. And so we've come here today to dramatize a shameful condition.

\section{西文示例(西里爾字母)}

(以下內容摘自\href{http://ru.tingroom.com/wap/index.php?moduleid=28&itemid=21380}{俄语学习网})

\textrussian{
  Я учусь в Педагогическом институте на факультете русского языка. В нашей группе двадцать студентов. Двое из них приехали из России. Наш коллектив очень дружный.

  (無衬綫體) \textsf{Каждый день студенческая жизнь начинается с шести часов утра . Мы со своими товарищами делаем утреннюю зарядку на площадке. После завтрака я иду на занятия. Во время обеденного перерыва я отдахаю в общежитии. Вечером я занимаюсь самостоятельно в аудитории, в 10 часов ложусь спать.}

  (等寬體) \texttt{По субботам и воскресеньям большую} часть времени я занимаюсь в библиотеке. (粗體) \textbf{Здесь бывает много студентов, всегда тихо и чисто.} (斜體) \textit{Я люблю читать журнал и газету. Вечером я иногда хожу в кино.} В клубе всегда показывают интересные фильмы, и китайские, и иностранные. Иногда мы вместе с друзьями гуляем на свежем воздухе и разговариваем о нашей студенческой жизни. У нас в институте много деревьев, ещё есть удобные скамейки и извилистые аллеи. Молодые .пары любят здесь  проводить время.
}

\end{document}

%%% Local Variables:
%%% mode: latex
%%% TeX-master: t
%%% End:
